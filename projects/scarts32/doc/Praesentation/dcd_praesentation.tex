%% LaTeX Beamer presentation template (requires beamer package)
%% see http://bitbucket.org/rivanvx/beamer/wiki/Home
%% idea contributed by H. Turgut Uyar
%% template based on a template by Till Tantau
%% this template is still evolving - it might differ in future releases!

\documentclass[compress]{beamer}

\graphicspath{{images/}}

\usepackage[ngerman]{babel}
\usepackage{graphicx}
\usepackage{epstopdf}
\usepackage{ifthen}

% font definitions, try \usepackage{ae} instead of the following
% three lines if you don't like this look
\usepackage{mathptmx}
\usepackage[scaled=.90]{helvet}
\usepackage{courier}

\usepackage[T1]{fontenc}

\hypersetup
{
    bookmarksnumbered = true,
    pdfauthor = Johannes Natter,
    pdftitle = DCD,
    pdfsubject = ,
    pdfkeywords = ,
    pdfcreator = ,
    pdfproducer = ,
    colorlinks = false,
    linkcolor = black,
    citecolor = black,
    filecolor = black,
    urlcolor = black
}

\mode<presentation>
{
\usetheme{default}

%    AnnArbor | Antibes | Bergen |
%    Berkeley | Berlin | Boadilla |
%    boxes | CambridgeUS | Copenhagen |
%    Darmstadt | default | Dresden |
%    Frankfurt | Goettingen | Hannover |
%    Ilmenau | JuanLesPins | Luebeck |
%    Madrid | Malmoe | Malmoe |
%    Montpellier | PaloAlto | Pittsburgh |
%    Rochester | Singapore | Szeged |
%    Warsaw | Amsterdam

\useoutertheme[footline=authortitle]{miniframes}

%    default | infolines | miniframes |
%    shadow | sidebar | smoothbars |
%    smoothtree | split | tree

\useinnertheme{circles}

%    circles | default | inmargin |
%    rectangles | rounded

\usecolortheme{whale}

%    albatross | beaver | beetle |
%    crane | default | dolphin |
%    dove | fly | lily | orchid |
%    rose |seagull | seahorse |
%    sidebartab | structure |
%    whale | wolverine

\definecolor{beamer@blendedblue}{rgb}{0.137,0.466,0.741}

\setbeamercolor{structure}{fg=beamer@blendedblue}
\setbeamercolor{titlelike}{parent=structure}
\setbeamercolor{frametitle}{fg=black}
\setbeamercolor{title}{fg=black}
\setbeamercolor{item}{fg=black}

\usefonttheme{structuresmallcapsserif}

%    default | professionalfonts | serif |
%    structurebold | structureitalicserif |
%    structuresmallcapsserif

\setbeamercovered{transparent}

\beamertemplatenavigationsymbolsempty
}

\title{Distributed Computing Device}

%\subtitle{}

% - Use the \inst{?} command only if the authors have different
%   affiliation.
%\author{F.~Author\inst{1} \and S.~Another\inst{2}}
\author{Johannes Natter}

% - Use the \inst command only if there are several affiliations.
% - Keep it simple, no one is interested in your street address.
\institute
{
\includegraphics[width=1.0cm]{./fhooe} \\
University of Applied Sciences \\
Upper Austria
}

\date{Sept. \ \the\year}

% This is only inserted into the PDF information catalog. Can be left
% out.
\subject{Distributed Computing Device}

% If you have a file called "university-logo-filename.xxx", where xxx
% is a graphic format that can be processed by latex or pdflatex,
% resp., then you can add a logo as follows:

%\pgfdeclareimage[height=0.5cm]{university-logo}{university-logo-filename}
%\logo{\pgfuseimage{university-logo}}

%% Kommandos
%---------------------------------------------------------------------------------------------------------------
\newlength{\pw}
\setlength{\pw}{1.0 \textwidth}

\newcommand{\todo}[1]{\textbf{\textcolor{red}{TODO: #1}}}

\newcommand{\ppic}[3][\empty]
{
    \begin{figure}[!ht]
        \centering
        \ifthenelse{\equal{#1}{\empty}}
        {\includegraphics[width=\pw]{./#3}}
        {\includegraphics[width=#1]{./#3}}
        \label{#2}
    \end{figure}
}

\newcommand{\sca}{\emph{SCARTS} }
\newcommand{\app}{\emph{Applikation} }

%---------------------------------------------------------------------------------------------------------------
%% END Kommandos

\begin{document}

{ % Titlepage
    \definecolor{head0}{rgb}{0.06,0.23,0.37}
    \definecolor{head1}{rgb}{0.10,0.35,0.55}
    
    \setbeamertemplate{headline}
    {{\color{head0}\rule[0mm]{\paperwidth}{0.54cm}}
     {\color{head1}\rule[0mm]{\paperwidth}{0.34cm}}}%
    
    \setbeamertemplate{footline}
    {{\color{head0}\rule[0mm]{\paperwidth}{0.34cm}}}%
    
    \begin{frame}
    \titlepage
    \end{frame}
}

\section*{Zielsetzung}
\begin{frame}{Zielsetzung}
    \begin{itemize}
      \item Computersystem f�r verteiltes Rechnen
      \item Angepasst durch programmierbare \\ Hard- und Software
      \item HW u. SW austauschbar
      \item Erweiterbar
    \end{itemize}
\end{frame}

\section*{�bersicht}
\begin{frame}{�bersicht}
\tableofcontents
\end{frame}

\section{Implementierung}
\subsection{Kommunikationspartner}
\begin{frame}
    \ \\

    \begin{tabular}{p{4.3cm}p{5.0cm}}
        \begin{itemize}
          \item Applikation und \emph{BSL} (bootstrap loader) \\
          \item Ein Programm aktiv
          \item BSL-Protokoll
          \begin{itemize}
            \item Speicher lesen/schreiben
            \item Version lesen
            \item Programmwechsel
          \end{itemize}
        \end{itemize}
        &
        \ \newline \ \newline \ \newline
        \includegraphics[width=0.5 \pw]{./kommunikationspartner}
    \end{tabular}
\end{frame}

\subsection{Hardware-Aufbau}
\begin{frame}
    \ppic[0.46 \pw]{concept_de0nano}{concept_de0nano}
\end{frame}

\subsection{Bootkonzept}
\begin{frame}
    \ppic[0.82 \pw]{boot_blank}{boot_blank}
\end{frame}

\begin{frame}
    \ppic[0.82 \pw]{boot_usb}{boot_usb}
\end{frame}

\begin{frame}
    \ppic[0.82 \pw]{boot_bsl}{boot_bsl}
\end{frame}

\begin{frame}
    \ppic[0.82 \pw]{boot_bsl_started}{boot_bsl_started}
\end{frame}

\begin{frame}
    \ppic[0.82 \pw]{boot_bsl_copy}{boot_bsl_copy}
\end{frame}

\begin{frame}
    \ppic[0.82 \pw]{boot_bsl_data}{boot_bsl_data}
\end{frame}

\begin{frame}
    \ppic[0.82 \pw]{boot_app}{boot_app}
\end{frame}

\begin{frame}
    \ppic[0.82 \pw]{boot_app_started}{boot_app_started}
\end{frame}

\begin{frame}
    \ppic[0.82 \pw]{boot_app_copy}{boot_app_copy}
\end{frame}

\begin{frame}
    \ppic[0.82 \pw]{boot_app_data}{boot_app_data}
\end{frame}

\section{Zeitplan}
\begin{frame}

\begin{table}[h!]
    \centering
    \begin{tiny}
        \begin{tabular}[t]{p{0.185 \paperwidth}p{0.150 \paperwidth}p{0.180
        \paperwidth}p{0.185 \paperwidth}} KW19 & KW20 & KW21 & KW22 \\
            \hline \\
            \emph{SCARTS-GCC} f�r \newline Windows 7 kompilieren &
            \emph{SCARTS-GCC} f�r \newline Windows 7 kompilieren &
            \emph{SCARTS}-Toolchain einrichten (Eclipse, Makefile,
            Linkerscript, \ldots) &
            DE0-Nano-Testbed mit \newline \emph{SCARTS}-Prozessor (VHDL) \\
            \ & \ & \ & \ \\
            \ & \ & \ & \ \\
            KW23 & KW24 & KW25 & KW26 \\
            \hline \\
            Leeres \emph{SCARTS}-C-Projekt, HEX-File-Konverter-Tool
            (C++) & GPIO- u. SPI-Modul \newline (VHDL) &
            Portieren von \emph{BSL} (C) &
            Treiber f�r \newline Flash-Speicher (C) \\
            \ & \ & \ & \ \\
            \ & \ & \ & \ \\
            KW27 & KW28 & KW29 & KW30 \\
            \hline \\
            Treiber f�r \newline Konfigurations\-speicher (C) &
            SDRAM-Controller (VHDL) &
            SDRAM-Controller (VHDL), \newline Boot-Programm (ASM), \newline
            Startup-Code (ASM) & Portieren von TCP/IP-Stack (C), Z�hlermodul (VHDL)
        \end{tabular}
    \end{tiny}
    \label{taboverviewkw}
\end{table}
\end{frame}

\section{Ergebnisse}
\begin{frame}
\begin{table}[h!]
    \centering
    \begin{tiny}
        \begin{tabular}[t]{p{6.5cm}r}
            Software & \ \\
            \hline
            \ & \ \\
            Prozessor & \emph{SCARTS32} \\
            Prozessortakt & 25MHz \\
            Gr�"se des Datenspeichers & 32KB \\
            Maximale Programmgr�"se der \app & 512KB \\
            Ben�tigte Bootzeit & ca. 1s \\
            \ & \ \\
            Hardware & \ \\
            \hline
            \ & \ \\
            Anzahl der ben�tigten Logikeinheiten f�r das \emph{User-Design} mit
            \emph{SCARTS}- \newline Prozessor, SDRAM-Controller, GPIO-, SPI-,
            ALTREMOTE- und Z�hler-Modul & 6.734 von 22.320
            \\
            Anzahl der ben�tigten Speicherbits f�r das \emph{User-Design} &
            350.208 von 608.256 \\
            Taktversorgung des FPGA's & 50MHz \\
            Netzwerkanschluss & 10Base-T Ethernet
        \end{tabular}
    \end{tiny}
    \label{taboverviewdcd}
\end{table}
\end{frame}

\end{document}
