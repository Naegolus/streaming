
\chapter{Kurzfassung}

F�r die Berechnung komplexer Probleme in vielen Bereichen der
Wissenschaft wird aktuell die Rechenleistung verschiedenster
Computersysteme genutzt. Solche universell ausgelegten Systeme sind jedoch
zumeist nicht f�r spezifischere Probleme angepasst.

Diese Arbeit behandelt die Entwicklung eines erweiterbaren Computersystems,
welches mit Hilfe eines FPGA-(Field Programmable Gate Array) Bausteins
einerseits die M�glichkeit bereitstellt, mit programmierbaren Hard- und
Softwarekomponenten das System f�r eine Aufgabe zu spezialisieren, andererseits
soll dieses Gemisch aus Hard- und Software durch ein Framework schnell und
flexibel austauschbar sein.

Das Ergebnis dieser Arbeit ist ein Konzept, sowie die Hard- und
Software des vielseitig einsetzbaren Computersystems. Die Aktualisierung und
der Austausch des Systems �ber das Internet ist durchf�hrbar. Die
Hardwaremodule k�nnen jedoch derzeit nicht, wie in dieser Arbeit gefordert, per
Software aktiviert werden, sondern ben�tigen das Aus- und Einschalten des Ger�tes.
