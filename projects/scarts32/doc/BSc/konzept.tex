\chapter{Entwicklung}
\label{cha:Konzept}

Dieses Kapitel widmet sich den gestellten Anforderungen an das \emph{DCD}. Es
wird hierzu aus Sicht des Benutzers vorgegangen, also die Verwendung des \dcd in
den Vordergrund gestellt. Zus�tzlich enth�lt dieses Kapitel die Auswahl der
Hardwarekomponenten. Die
ausf�hrliche Beschreibung der gew�hlten Hardwarekomponenten erfolgt in
Kapitel~\ref{cha:Hardware}.

\section{Kommunikation}

Zu Beginn stellt sich die Frage, welche Parteien an der Kommunikation
beteiligt sind und was �bertragen werden soll.
Das \dcd soll die Arbeitspakete eines Projektservers empfangen, diese
verarbeiten und die Ergebnisse an den Projektserver zur Validierung zur�ck
senden. Diese Aufgabe wird von einem Programm am \emph{DCD}, im
folgenden als \app bezeichnet, bearbeitet.

Zus�tzlich kann der Benutzer jederzeit die aktuelle \app durch eine
andere ersetzen. Dies ist die Aufgabe eines weiteren Programmes am \emph{DCD},
welches in diesem Dokument \emph{BSL} (\emph{bootstrap loader}) genannt wird.
Bevor das \dcd eine andere \app empfangen kann, muss der \bsl gestartet werden.
Dieser kann von einem Benutzerprogramm per Internet eine neue \app empfangen,
am \dcd speichern und anschlie�end starten.

Die \app kann nun mit dem selben Projektserver
kommunizieren oder auch mit einem anderen und f�r ein anderes Projekt rechnen.
Die Standorte aller beteiligten Parteien k�nnen unterschiedlich sein.
\abb{fig:kommu} zeigt die Kommunikationspartner.

\pic{fig:kommu}{Kommunikationspartner}{Kommunikationspartner}

\pagebreak

\subsection{Protokolle}

Die �bertragung der Daten aller Kommunikationspartner erfolgt �ber das Internet.
\dcd und Benutzerprogramm ben�tigen daher einen entsprechenden
Protokollstack. F�r die �bertragung der \app steht ein, vom Autor dieses
Dokumentes spezifiziertes, Protokoll auf Ethernetebene zur Verf�gung.
Das Benutzerprogramm ist ebenfalls vorhanden. Kapitel~\ref{cha:Software}
beschreibt Benutzerprogramm und Protokoll im Detail. F�r die Kommunikation �ber
das Internet muss das Benutzerprogramm erweitert und das Internetprotokoll mit
Transportschicht in den Protokollstack integriert werden. Als Transportschicht
wird das \emph{User Datagram Protocol (UDP)} gew�hlt, da es eine
geeignete einfache Daten�berpr�fung besitzt und weniger Resourcen im \dcd
beansprucht als die Implementierung des \emph{Transmission Control Protocol's
(TCP)}. Die Erweiterung ist in \abb{fig:proto_app} dargestellt.
Die verwendeten Protokolle zwischen Projektserver und \dcd sind
projektspezifisch.

\pic{fig:proto_app}{Erweiterung des
Benutzerprogramms und dessen Protokollstacks}{proto_app}

\subsection{Arbeitsablauf}

Die \app wird vom Benutzer an das \dcd gesendet und aktiviert.
Sobald diese startet, wird eine Verbindung mit dem Projektserver aufgebaut und
ein Arbeitspaket angefordert. Nach erfolgter Verarbeitung der Daten, schickt
das \dcd die Ergebnisse zur Beurteilung an den Server zur�ck und fordert
sogleich neue Aufgaben an. Alle Ergebnisse werden vom Projektserver auf
Richtigkeit �berpr�ft und die Resultate an das \dcd gesendet.

Eventuelle Statusmeldungen, Statistiken und Arbeitsprotokolle k�nnen
applikationsspezifisch vorhanden sein. Dies soll jedoch nicht Teil dieser
Arbeit sein.


\section{Anforderungen}

\subsection{Hardware}

Folgende Punkte sind aus Hardwaresicht vom \dcd zu erf�llen:
\begin{itemize}
  \item M�glichst Kompakt,
  \item leicht Erweiterbar,
  \item arbeitet autonom
\end{itemize}
Zur Erf�llung dieser Aufgaben wird eine Kombination aus Prozessor und
FPGA verwendet. Hierbei kann zum einen die gesamte Netzwerkkommunikation
komfortabel in C implementiert werden, zum anderen k�nnen rechenintensive und
zeitkritische Aufgaben vom FPGA �bernommen werden. Aus Sicht des Prozessors
sind diese unterst�tzenden Hardwaremodule memory-mapped als
\emph{Spezialregister} verf�gbar.

Um das Rechenger�t kompakt zu halten, wird ein Soft-Core Prozessor verwendet.
Hierbei handelt es sich um einen Prozessor, welcher in einem FPGA-Design
integrierbar ist. Somit kann das gesamte Framework inklusive \bsl und \app auf
einem Chip bzw. Entwicklungsboard ausgef�hrt werden.

Als Entwicklungsboard wird das \emph{DE0-Nano Development and Education
Board}\footnote{DE0-Nano Board,
http://www.terasic.com.tw/cgi-bin/page/archive.pl?No=593} der Firmwa Terasic verwendet.

\subsection{Verwendung (Toolchain)}

\begin{itemize}
  \item Die \app besteht aus einer Datei am PC und beinhaltet alle, f�r
  die Erf�llung der Aufgaben erforderlichen, Programme und Hardwaremodule 
  \item Die \app kann per
  Doppelklick auf die Datei, oder per Konsole an das Ger�t gesendet und
  aktiviert werden
\end{itemize}

\section{�bersicht}

\abb{concept} zeigt den Aufbau des \emph{DCD's}. Das Cyclone-IV-FPGA,
der SDRAM und der Konfigurationsspeicher des DE0-Nano Board's werden verwendet.
Erweitert wird dieses durch einen Flash Speicher und einen Netzwerk-Controller. 
Der Flash Speicher enth�lt den \bsl und die \emph{Applikation},
welche beim Start in den SDRAM kopiert werden. Im Konfigurationsspeicher ist
das FPGA-Design gespeichert. Dieses beinhaltet den Soft-Core-Prozessor mit
SDRAM-Controller und alle Hardwaremodule inklusive applikationsspezifischem
Spezialmodul und Co-Prozessor Modul, falls diese f�r die \app ben�tigt werden.

\pic{concept}{\emph{DCD}-�bersicht}{concept_de0nano}
