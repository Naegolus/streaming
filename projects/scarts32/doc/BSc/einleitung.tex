\chapter{Einleitung}
\label{cha:Einleitung}

Die Verwendung der ungenutzten Rechenleistung vieler verteilter PCs bzw. deren
Grafikprozessoren und sogar der Einsatz von Spielkonsolen unterst�tzt inzwischen
einige rechenintensive Projekte in den unterschiedlichsten Bereichen der
Forschung (Mathematik, Physik, Chemie, Medizin, Finanzwesen, usw\ldots) mit
kostenloser Rechenleistung.
Das L�sen von Berechnungsaufgaben durch verteilte
Computersysteme wird als verteiltes Rechnen bezeichnet\footnote{Verteiltes Rechnen, \url{http://de.wikipedia.org/wiki/Verteiltes_Rechnen}}. Durch die bis
zu mehreren hunderttausend vernetzten Computersysteme k�nnen diese Projekte auf
eine Rechenkraft �quivalent aktueller Supercomputer
zur�ckgreifen \cite{Folding12}.

Im Rahmen dieser Bachelorarbeit soll ein netzwerkf�higes Ger�t, im folgenden
\dcd (Distributed Computing Device) genannt, zur Unterst�tzung solcher Projekte
im Internet entstehen.
Dieses soll nicht an ein bestimmtes Projekt gebunden sein, sondern durch eine leicht austauschbare
\app f�r unterschiedliche Projekte arbeiten k�nnen. Zu Beginn dieser
Arbeit wird schrittweise ein Konzept (Kap.~\ref{cha:Konzept}) f�r dieses
Vorhaben erstellt und jeweils die Motivation der Entscheidungen dokumentiert.
Kapitel~\ref{cha:Hardware} beschreibt ausf�hrlich die ausgew�hlten
Hardwaremodule. Der Aufbau der Software und die Verwendung der Toolchain werden
in Kapitel~\ref{cha:Software} erl�utert. 
Kapitel~\ref{cha:Schluss} untersucht die Verwendbarkeit des \dcd
anhand des Beispielprojektes \emph{Bitcoin}, fasst die Eigenschaften des
Ger�tes zusammen und listet m�gliche Erweiterungen der Hard- und Software des
Computersystems auf.

\pagebreak

\section{Motivation}
\label{sec:motivation}

Des Systems des verteilten Rechnens bedienen sich mittlerweile viele Projekte im
Internet. Einige davon sind in der \emph{BOINC}\footnote{Berkeley Open
Infrastructure for Network Computing, \\
\url{http://de.wikipedia.org/wiki/Berkeley_Open_Infrastructure_for_Network_Computing}}
organisatorisch zusammengefasst. \emph{BOINC} ist ein Framework f�r Projekte,
welche verteiltes Rechnen nutzen m�chten.

Das Projekt \emph{Bitcoin}\footnote{Bitcoin,
\url{http://de.wikipedia.org/wiki/Bitcoin}} bedient sich im Gegensatz dazu
einer eigenen Infrastruktur. \emph{Bitcoin} ist ein dezentral
organisiertes, elektronisches Geld, welches aktuell (Stand August 2012) einen
Wechselkurs von 8,49 Euro \cite{BitcoinKurs12} hat.

Die Informationen einer get�tigten Transaktion zwischen zwei Bitcoin Benutzern
werden vom jeweiligen Sender an alle beteiligten Nutzer des dezentralen
Netzwerkes gesendet. Diese sammeln alle empfangenen und im Netzwerk als unbest�tigt
geltenden Transaktionen in einer Liste, welche als Block bezeichnet wird.
Ein beliebiger Benutzer kann seinen Block abschlie�en und somit die darin
enthaltenen Transaktionen vom Netzwerk als best�tigt markieren lassen.
Zum Abschlie�en eines Blockes muss der SHA256\footnote{Secure Hash Algorithm,
http://de.wikipedia.org/wiki/SHA256}-Hashwert des Blockheaders berechnet
werden.
Der Header enth�lt unter anderem einen Zeitstempel, den f�r das Abschlie�en des
Blockes verantwortlichen Benutzer und einen 32 Bit langen, variablen Wert
(Nonce).
Anschlie�en wird der SHA256-Hashwert des erhaltenen Hashwertes vom Blockheader
berechnet. Ist der Wert dieses Ergebnisses nun kleiner-gleich als der eines, vom
Netzwerk bestimmten, Zielwertes, so ist der Block erfolgreich abgeschlossen.
Ist der Wert des Ergebnisses gr��er, muss der Benutzer den variablen Wert des
Blockheaders um eins erh�hen und den SHA256 Hashwert erneut zweimal berechnen.
Das L�sen dieses Problems kann, abh�ngig vom rechnenden Computersystem, mehrere
Minuten ben�tigen. Ob ein Block g�ltig ist, kann jedoch vom Netzwerk schnell
�berpr�ft werden\footnote{Einwegfunktion,
http://de.wikipedia.org/wiki/Einwegfunktion}. F�r das Abschlie�en eine Blockes
erh�lt der verantwortliche Benutzer von Netzwerk als Anerkennung Bitcoins, da
die Benutzer des Bicoin-Netzwerkes dadurch die get�tigten Transaktionen
nachvollziehen k�nnen.

Die Berechnung der Bitcoin-Hashes dient als exemplarische Aufgabe zum Einsatz
des \emph{DCD}.
Unabh�ngig von der vorhandenen Infrastruktur und den verwendeten Protokollen zur
Kommunikation soll das \dcd auch f�r andere Projekte arbeiten k�nnen.
